\documentclass[12pt]{article}
\usepackage{amsmath, amssymb}
\usepackage{verbatim}

\newcommand*\inv{^{-1}}
\DeclareMathOperator\rank{rank}

\title{Factory Planning Algorithm}
\author{Alex Loomis}

\begin{document}
\pagenumbering{gobble}
\maketitle

Given a collection of recipes,
create a matrix $M$ where each column of $M$ is a recipe,
and each row corresponds to an item.
For instance, if mining creates two iron per second
and smelting consumes five iron to create one steel,
\[
  M = \begin{pmatrix}
    2 & -5 \\
    0 & 1 \\
  \end{pmatrix}.
\]

Multiplying by a column of recipes, $R$,
yields a column of outputs produced, $P$.
Mining two times and smelting once produces
one steel with a deficit of one iron:
\[
  R = \begin{pmatrix} 2 \\ 1 \end{pmatrix}, \quad
  P = \begin{pmatrix} -1 \\ 1 \end{pmatrix}, \quad
  MR = P.
\]
The user can specify up to $\rank M$ values of $P$, however,
they must be chosen such that the submatrix of corresponding rows
has linearly independent rows.
If they specify fewer, the remaining choices can be assumed to be zero,
to maximize self-sufficiency of the production line.
Our goal then is to solve for $R$.

If the rank of $M$ is less than the number of columns of $M$
then several recipes are linearly dependent,
so we can choose $\rank M$ linearly independent columns.
The system now has a unique solution, so we solve.

This has been implemented in Haskell.
As a test input, try
\begin{verbatim}
Matrix:
[[1,-5],[0,1],[-1,-6]]
Constraints:
[(1, Just 1)]

[([0,1],(2><1)
 [ 5.0
 , 1.0 ]),([0,1],(2><1)
 [ -6.0
 ,  1.0 ])]
\end{verbatim}
This means that if we have the two recipies;
one iron from one coal,
and one steel from five iron and six coal,
and we want one steel,
we can do so with minimal inputs and byproducts by
either mining five times and smelting once,
or by mining negative six times and smelting once.

\end{document}
